\documentclass[12pt,a4paper]{article}
\usepackage{ctex}
\usepackage{amsmath,amscd,amsbsy,amssymb,latexsym,url,bm,amsthm}
\usepackage{epsfig,graphicx,subfigure}
\usepackage{enumitem,balance}
\usepackage{wrapfig}
\usepackage{mathrsfs,euscript}
\usepackage[usenames]{xcolor}
\usepackage{hyperref}
\usepackage[vlined,ruled,linesnumbered]{algorithm2e}
\hypersetup{colorlinks=true,linkcolor=black}

\newtheorem{theorem}{Theorem}
\newtheorem{lemma}[theorem]{Lemma}
\newtheorem{proposition}[theorem]{Proposition}
\newtheorem{corollary}[theorem]{Corollary}
\newtheorem{exercise}{Exercise}
\newtheorem*{solution}{Solution}
\newtheorem{definition}{Definition}
\theoremstyle{definition}

\renewcommand{\thefootnote}{\fnsymbol{footnote}}

\newcommand{\postscript}[2]
 {\setlength{\epsfxsize}{#2\hsize}
  \centerline{\epsfbox{#1}}}

\renewcommand{\baselinestretch}{1.0}

\setlength{\oddsidemargin}{-0.365in}
\setlength{\evensidemargin}{-0.365in}
\setlength{\topmargin}{-0.3in}
\setlength{\headheight}{0in}
\setlength{\headsep}{0in}
\setlength{\textheight}{10.1in}
\setlength{\textwidth}{7in}
\makeatletter \renewenvironment{proof}[1][Proof] {\par\pushQED{\qed}\normalfont\topsep6\p@\@plus6\p@\relax\trivlist\item[\hskip\labelsep\bfseries#1\@addpunct{.}]\ignorespaces}{\popQED\endtrivlist\@endpefalse} \makeatother
\makeatletter
\renewenvironment{solution}[1][Solution] {\par\pushQED{\qed}\normalfont\topsep6\p@\@plus6\p@\relax\trivlist\item[\hskip\labelsep\bfseries#1\@addpunct{.}]\ignorespaces}{\popQED\endtrivlist\@endpefalse} \makeatother

\begin{document}
\noindent

%========================================================================
\noindent\framebox[\linewidth]{\shortstack[c]{
\Large{\textbf{Lab00-Proof}}\vspace{1mm}\\
CS214-Algorithm and Complexity, Xiaofeng Gao, Spring 2020.}}
\begin{center}
\footnotesize{\color{red}$*$ If there is any problem, please contact TA Yiming Liu.}

% Please write down your name, student id and email.
\footnotesize{\color{blue}$*$ Name:Haotian Xue  \quad Student ID:518021910506 \quad Email: xavihart@sjtu.edu.cn}
\end{center}

\begin{enumerate}
    \item
    Prove that for any integer $n>2$, there is a prime $p$ satisfying $n<p<n!$. {\color{blue}(Hint: consider a prime factor $p$ of $n!-1$ and prove by contradiction)}
    \begin{proof}
        Assume there is no prime in range $(n, n!)$. Then for any number $x$ in $(n, n!)$, the prime factor of $x$ is no more than $n$, to be specific, the prime factor
        of $x$ are all in $\{2, 3, 4, ..., n\}$. 
        
        However, for the number $n! - 1$, it is not divisible by any number in $\{2, 3, 4, ..., n\}$, 
        which contradicts the assumption.


        Therefore, there is a prime $p$ satisfying $n<p<n!$.
    \end{proof}

    \item
    Use the minimal counterexample principle to prove that for any integer $n>17$, there exist integers $i_n\ge 0$ and $j_n\ge 0$, such that $n = i_n \times 4 + j_n \times 7$.
    \begin{proof}
      $P(n)$ : there exist integers $i_n\ge 0$ and $j_n\ge 0$, such that $n = i_n \times 4 + j_n \times 7$. 
      
      If $P(n)$ is not true for every $n > 17$, then there must be a smallest $k$, such that $P(k)$ is false.
      
      Since $18 = 4 \times 1 + 7 \times 2$, we have $k > 18$, $k - 1 > 17$.

      Since $k$ is the smallest value to make $P(k)$ false, then $P(k - 1)$ is true: 
      
      \begin{equation*}
          k - 1 = i_{k - 1} \times 4 + j_{k - 1} \times 7 
      \end{equation*}
      
      If $k - 1 \equiv 0(mod 4)$, $j_{k - 1} = 0$, $i_{k - 1} \ge 5$ for $k > 17$, in this case:

      \begin{equation*}
        k = (i_{k - 1} - 5) \times 4 + 3 \times 7
    \end{equation*}

    $i_k = i_{k - 1} - 5 \ge 0$, $j_k = 3$ 
    
    If $k - 1$ can not be divided by $4$, then $j_{k - 1} >= 1$, in this case:

    \begin{equation*}
        k = (i_{k - 1} + 2) \times 4 + (j_{k - 1} - 1) \times 7 
    \end{equation*}

    $i_k = i_{k - 1} + 2 \ge 0$, $j_k = j_{k - 1} - 1 \ge 0$

    So for both cases, $P(k)$ is true. We have derived a contradiction,
    which allows us to conclude that our original assumption is false.


      
    \end{proof}

    \item
    Let $P=\{p_1, p_2, \cdots\}$ the set of all primes. Suppose that $\{p_i\}$ is monotonically    increasing, i.e., $p_1=2$, $p_2=3$, $p_3=5$, $\cdots$. Please prove: $p_n<2^{2^n}$. {\color{blue}(Hint: $p_i \nmid (1+\prod_{j=1}^n p_j), i=1,2,\cdots,n$.)}
    \begin{proof}
       We will use strong Mathematical Induction to solve the problem, we define $P(n)$: $p_n < 2^{2^{n}}$
 
       \textbf{Basis step}: $p_1 = 2 < 4$.

       \textbf{Induction hypothesis}: If $P(i)$ is true for $1 \le i < k$, then $P(k)$ is true$(k > 1)$.

       \textbf{Proof ot induction step}: Since $p_k$ id the $k$th prime number, for any integer $x \in [p_{k - 1}, p_{k})$, the prime factors of $x$ are in 
       ${p_1, p_2, ... , p_{k - 1}}$. Since number $\prod\limits_{i=1}^{k - 1}{p_{i}} + 1$ cannot be divided by any prime number in 
       ${p_1, p_2, ... , p_{k - 1}}$, we can get:

       \begin{equation*}
           p_k \le \prod_{i=1}^{k - 1}{p_i} + 1
       \end{equation*}

       According the the induction hypothesis, $p_{i} < 2^{2^{i}}$ for $i=1, 2, ..., k - 1$. 
       \begin{equation*}
        p_k < \prod_{i=1}^{k - 1}{2^{2^{i}}} + 1 = 2^{2^{k - 2}} + 1 < 2^{2^{k}} (k > 1)
       \end{equation*}

       Therefore, the statement is true.
   
       



    \end{proof}

    \item
    Prove that a plane divided by $n$ lines can be colored with only $2$ colors, and the adjacent regions have different colors.
    \begin{proof}
        We use Mathematical Induction to solve the problem.
        
        \textbf{Basis step}: When $n = 1$, we just paint the two section of plane different colors.
        
        \textbf{Induction hypothesis}: If the statement is true for $n = k$, it is true for $n = k + 1$.
        
        \textbf{Proof of induction step}: For $n = k + 1$, we just pick out one line, so the plane now is divided by
        $k$ lines. According to the induction hypothesis, the plane can be coloured in only two colors and the adjacent
        regions have different colors.

        Then we add back the line we just picked out (name it $l$) and keep the colored unchanged. Considering the two side of $l$:
        $R_1 = \{r^{1}_{1}, r^{1}_{2}, ..., r^{1}_{m}\}$, $R_2 = \{r^{2}_{1}, r^{2}_{2}, ..., r^{2}_{n}\}$, just reverse the color of regions
        in $R_1$, it is obvious that this way of painting satisfies the requirement, because for any adjacent regions $r_1$ and $r_2$:

        (1) If $r_1$ and $r_2$ are in the same side of $l$, their colors are different both after and before being reversed.

        (2) If $r_1$ and $r_2$ are in different sides of $l$, we can get that $r_1$ and $r_2$ belong to the same region before line $l$ was added.
        So the colors of $r_1$ and $r_2$ are different after the reversion operation.

        Therefore the statement is true.
    \end{proof}

\end{enumerate}

\vspace{20pt}

\textbf{Remark:} You need to include your .pdf and .tex files in your uploaded .rar or .zip file.

%========================================================================
\end{document}
